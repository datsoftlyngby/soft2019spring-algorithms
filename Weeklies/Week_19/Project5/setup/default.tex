\usepackage[english]{babel}
\usepackage[T1]{fontenc}        % Support for fonts with ������������������ and other foreign characters.
\usepackage[utf8]{inputenc}     % Support for UTF-8 encoded input documents
\usepackage{CJKutf8}
%\usepackage{fullpage}
\usepackage{graphicx}           % Support for including graphics as png, gif, and jpeg
\usepackage{amssymb}            % Support for alterantive symbols
\usepackage{amsmath}            % Support for mathematical symbols
\usepackage{stmaryrd}                 % Support for \varodot and other symbols
\usepackage{hyperref}
\usepackage{cleveref}                 % For extended reference support
\usepackage{mathrsfs}
%\usepackage{bbding}
\usepackage{listliketab}        % Support for tabulated lists
%\usepackage{enumitem}           % Support for indented description items and more
\usepackage{xcolor}              % Support for colored text
\usepackage{listings}           % Support for code listings
\usepackage{nameref}            % Enables refernces to names.
%\usepackage{makeidx}            % For creating indexes
\usepackage{wasysym}            % For symbols as \smiley
\usepackage{hyperref}    		% For using URLs
\usepackage{arydshln}           % dashed lines in tables
%\usepackage{mdframed}
%\usepackage{multirow}			% for multirow tables
\usepackage{tikz}               % For doing vector graphics
\usetikzlibrary{shapes.multipart}
\usepackage{textpos}
\newcommand{\hdline}{\hdashline[1pt/3pt]}
\newcommand{\st}{\textsuperscript{st} }
\newcommand{\nd}{\textsuperscript{nd} }
\newcommand{\rd}{\textsuperscript{rd} }
\newcommand{\xth}{\textsuperscript{th} }
\newcommand{\cool}[1]{\textcolor{red!70!black}{\textbf{#1}}}
\newcommand{\code}[1]{\textcolor{green!50!black}{\texttt{#1}}}
\newcommand{\shadow}[1]{\textcolor{gray}{#1}}
\newcommand{\ditem}[2]{\item[#1]\ \\#2}

%\newcommand{\fact}[1]{\begin{center}\begin{Large}#1\end{Large}\end{center}}

\newcommand{\Z}{\mathbb{Z}}
\newcommand{\R}{\mathbb{R}}
\newcommand{\N}{\mathbb{N}}
\newcommand{\Q}{\mathbb{Q}}
\newcommand{\Rel}[3]{#1\,#2\;#3}
\newcommand{\RRel}[2]{#1\,R\;#2}
\newcommand{\ncomp}{\npreceq}
\newcommand{\divides}{\;|\;}
\newcommand{\where}{\,|\,}

\newcommand{\pSet}[1]{\mathscr{P}(#1)}
\newcommand{\E}{\exists}
\newcommand{\A}{\forall}

%\usepackage{color}
%\usepackage{perpage} %the perpage package
%\MakePerPage{footnote} %the perpage package command
\newcounter{exercisecounter}
\newenvironment{exercise}%
    {\refstepcounter{exercisecounter}}%
    {}

\definecolor{lightgray}{rgb}{.9,.9,.9}
\definecolor{darkgray}{rgb}{.4,.4,.4}
\definecolor{purple}{rgb}{0.65, 0.12, 0.82}
\definecolor{cphbg}{rgb}{0.004, 0.082, 0.227}
\definecolor{cphfg}{rgb}{0.984, 0.688, 0.250}

\lstdefinelanguage{JavaScript}{
    keywords={typeof, new, true, false, throw, catch, function, return, null, catch, switch, var, if, in, while, do, else, case, break},
    sensitive=true,
    comment=[l]{//},
    morecomment=[s]{/*}{*/},
    morestring=[b]',
    morestring=[b]"
	}
\lstdefinelanguage{Kotlin}{
    keywords={val, var, true, false, throw, catch, fun, return, null, catch, while, when, if, else, while, do, else, break, continue, in, sealed, class, data},
    sensitive=true,
    comment=[l]{//},
    morecomment=[s]{/*}{*/},
    morestring=[b]',
    morestring=[b]"
	}
\lstdefinelanguage{CSharp}{
    keywords={void, this, super, private, protected, public, new, true, false, throw, catch, finally, return, null, forall, for, var, if, in, while, do, else, case, break, bool, int, long, string, object, real, double, byte, char},
    sensitive=true,
    comment=[l]{//},
    morecomment=[s]{/*}{*/},
    morestring=[b]"
  }
\lstset{
	language=Java,
	frame=single,
	tabsize=2,
	escapeinside={(*}{*)},
	basicstyle=\footnotesize\ttfamily,
    keywordstyle=\bfseries\color{blue},
    commentstyle=\itshape\color{green!40!black},
    identifierstyle=\color{black},
    stringstyle=\color{orange}
	}
\lstnewenvironment{kotlin}
	{\lstset{
		language=Kotlin,
		backgroundcolor=\color[RGB]{240,240,255}
		}}
	{}
\lstnewenvironment{java}
	{\lstset{
		language=Java,
		backgroundcolor=\color[RGB]{240,240,255}
		}}
	{}
\lstnewenvironment{sql}
	{\lstset{
		language=SQL,
		backgroundcolor=\color[RGB]{255,240,240}
		}}
	{}
\lstnewenvironment{js}
	{\lstset{
		language=JavaScript,
		backgroundcolor=\color[RGB]{240,255,240}
		}}
	{}
\lstnewenvironment{cs}
  {\lstset{
    language=CSharp,
    backgroundcolor=\color[RGB]{240,255,240}
    }}
  {}
\lstnewenvironment{xml}
  {\lstset{
    language=XML,
    backgroundcolor=\color[RGB]{240,255,240}
    }}
  {}
%\pagestyle{empty}
