\documentclass[12pt,a4paper,final]{article}
\makeatletter
\def\input@path{{./setup/}{.}}
\makeatother
\usepackage{cph-document}
\usepackage{cph-listings}
\setlength{\parindent}{0cm}     % Default is 15pt.
\setlength{\parskip}{1em plus4mm minus3mm}


\makeindex

\title{Project \#5 -- Text Analysis Using Tries}
\author{Jacob Trier Frederiksen \& Anders Kalhauge}
\date{Algorithms and Datastructure, Spring 2019}


\begin{document}
\maketitle

Mantra: work with your groups! \\

%------------------------------------------------------
\textbf{THIS PART IS PROBLEM FORMULATION}: \\ ''We want to .... implement a Trie (from Weekly 18, Book Chapter 5.2) that works on the English alphabet, as well as \' ~.
... shall save information about each word in a text with the word and it's number of occurences. Try the data structure on Shakespeares complete works ...Write out all words with their counts ...''

%------------------------------------------------------
\textbf{THIS PART IS PROBLEM FORMULATION}: \\ Tips: below you can find a \underline{suggestion} for structuring your report on this assignment. This is not law, only suggestion.
\begin{description}
   \item[Contributions]: Theory on this topic has found that ... our project concludes that ... 
   \item[Background]: Analyzing ...  and ...  can result in valuable information ... which cannot be gained by directly ... 
   \item[Research Questions]: Can a ... be used to ... ? 
   \item[Methodology]: This project took the approach to compare ... with ... The analysis was done by choosing various ... as the observations. 
   \item[Findings]: We found that ... which is seems to ... We can therefore [support]/[challenge] theory ... from our studies in this project.
   \item[Outlook]: We think that ... for future investigations ... to obtain ... 
\end{description}

Upload to your group's report and solution to the \href{https://www.peergrade.io}{\textcolor{blue}{\underline{Peergrade website}}}, no later than Monday May 20$^\textrm{th}$, 22:30. 

\end{document}
